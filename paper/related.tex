\cite{zhang2008back} build recipe retrieval and creation systems leveraging the relations between food-related words in WordNet. Users can enter a list of ingredients they have, and get presented with suggested recipes that best match the input query. To do that, they also develop a similarity measure between two lists of ingredients. Furthermore, their system can accurately predict a type of dish and a type of cuisine.

\cite{teng2012recipe} build and examine ingredient-ingredient networks of complements and substitutes of ingredients. Networks were built using PMI-weighted co-occurences of ingredients in recipes, and suggestions for ingredient modifications in recipes, respectively. Ratings of recipes were predicted well, combining features from both models.


\cite{regneri2013grounding} build a corpus combining high-quality video recordings of cooking tasks with aligned textual descriptions, who in turn are aligned to pre-defined activity descriptions. While it also includes (a few) cooking ingredients, its focus is not on lexical and distributional knowledge of them, but on the cooking process itself, and the actions and tools involved, plus linking it with visual information. Similar work in semantic parsing of instructional cooking videos was done by \cite{malmaud2014cooking}.

In a similar direction goes \cite{tasse2008sour}, who develop a formal instruction language called \textit{MILK} that recipes can be written in, and develop a small corpus of 350 recipes translated (by annotators) into \textit{MILK}. In contrast, we build on a large and inconsistent web corpus and therefore don't model the cooking process.


\cite{ahn2011flavor} test Rozin's flavor principle \citep{rozin1973flavor} at the molecular level across the world, and find that Western European and North American recipes tend to pair ingredients that share the same flavor, whereas the East Asian recipes combine ingredients that do not have overlapping compounds. This difference might characterize a cuisine. Recently, the ``food pairing hypothesis'' was tested for Indian cuisine too \citep{2015arXiv150203815J}. It was found that the flavor sharing was significantly less than expected by chance and that the spices are ingredients that contribute the most to the negative food pairing. 
